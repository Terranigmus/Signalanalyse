\chapter{Stapelung von Signalen}

\subsubsection*{Beispiel} Überlagerung zweier Signale mit gleicher Frequenz $\omega_0$, aber unterschiedlicher Phase.  Das ist ein einfaches Beispiel für eine Stapelung mit $t_i\not= 0$ und $n_i=0$. Die zwei Signale sind:
\begin{align*}
s_1(t) & =cos(\omega_0 t+\varphi_1)\\
s_2(t) & =cos(\omega_0 t+\varphi_2).
\end{align*}
Mit dem Additionstheorem ergibt die Überlagerung:
\begin{equation}
s_1(t)+s_2(t)=2cos\left(\omega_0 t + \frac {\varphi_1 + \varphi_2}{2}\right)\,cos\left(\frac {\varphi_1 - \varphi_2}{2}\right).
\end{equation}
D.h. die Frequenz ändert sich nicht. Die Phasenverschiebung wird gemittelt und es ergibt sich ein zeitunabhängiger Wichtungsfaktor, der von der Phasendifferenz abhängt. Für $\varphi_1-\varphi_2= \pi $ löschen sich die Signale gegenseitig aus.\\\\
Für eine Phasenverschiebung von $|\varphi_1-\varphi_2|\le \frac{\pi}{2}$ ($\Delta\varphi = \frac{\lambda}{4}$) gilt für den Wichtungsfaktor $cos(\frac{\varphi_1-\varphi_2}{2})\ge 0.7071$ und es liegt konstruktive Interferenz vor.\\ 

\subsubsection*{Beispiel}
Überlagerung zweier Signale mit unterschiedlicher Frequenz:
\begin{align*}
s_1(t) & =cos(\omega_1 t)\\
s_2(t) &=cos(\omega_2 t)
\end{align*}
Mit dem Additionstheorem folgt:
\begin{equation}
s_1(t)+s_2(t)= 2 cos\left(\frac{\omega_1+\omega_2}{2}t\right)\,cos\left(\frac{\omega_1-\omega_2}{2}t\right).
\end{equation}

Es ergibt sich eine Schwebung mit einer hochfrequenten Trägerfrequenz $\frac{\omega_1+\omega_2}{2}$ und einer niederfrequenten Amplitudenmodulation mit der Frequenz $f_A = \frac{1}{2\pi}\frac{\omega_1-\omega_2}{2}$. Interessant ist, dass zu den Zeitpunkten $\frac{n}{2f_A}+\frac{1}{4f_A}, n \in G$ ein Phasensprung auftritt. Die Maxima der Einhüllenden sind ebenfalls $\frac{1}{2f_A}$ entfernt. Beachte: wird eine Fourieranalyse von diesem Signal gemacht, treten nur die Frequenzen $\omega_1$ und $\omega_2$ auf. Die Formulierung mit amplitudenmodulierter Trägerfrequenz ist jediglich äquivalent.\\ 

\subsubsection*{Beispiel}
Ein Signal soll zwei Frequenzen $\omega_1$ und $\omega_2$ enthalten: $s(t) = cos(\omega_1t) + cos(\omega_2t)$. Betrachtet wird die Überlagerung zweier zeitverschobener Signale:
\begin{align*}
s_1(t) &=s(t+t_1)\\
s_2(t) &=s(t+t_2)
\end{align*}

Mit dem Additionstheorem folgt:
\begin{multline}
s_1(t)+s_2(t)= 2 cos\left(\omega_1 t +\omega_1\frac{t_1+t_2}{2}\right)\,cos\left(\omega_1\frac{t_1-t_2}{2}\right)\\
+2 cos\left(\omega_2 t +\omega_2\frac{t_1+t_2}{2}\right)\,cos\left(\omega_2\frac{t_1-t_2}{2}\right).
\end{multline}

\begin{itemize}
\item Die Frequenzen $\omega_1$ und $\omega_2$ bleiben erhalten.
\item $cos(\omega_1\frac{t_1-t_2}{2})=f_1$ und $cos(\omega_2\frac{t_1-t_2}{2})=f_2$ sind zeitunabhängige Wichtungsfaktoren für die beiden Frequenzen.
\item Für $\omega_1<\omega_2$ ist $f_1>f_2$, d.h. die Stapelung ist ein Tiefpass. Die Bedingung für konstruktive Interferenz ist für tiefe Frequenzen eher erfüllt als für hohe Frequenzen. 
\end{itemize}