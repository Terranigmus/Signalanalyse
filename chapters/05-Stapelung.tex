\chapter{Stapelung von Signalen}

\section{Beispiele der Stapelung}
\subsubsection*{Beispiel}
Überlagerung zweier Signale mit gleicher Frequenz $\omega_0$, aber unterschiedlicher Phase.  Das ist ein einfaches Beispiel für eine Stapelung mit $t_i\not= 0$ und $n_i=0$. Die zwei Signale sind:
\begin{align*}
s_1(t) & =cos(\omega_0 t+\varphi_1)\\
s_2(t) & =cos(\omega_0 t+\varphi_2).
\end{align*}
Mit dem Additionstheorem ergibt die Überlagerung:
\begin{equation}
s_1(t)+s_2(t)=2cos\left(\omega_0 t + \frac {\varphi_1 + \varphi_2}{2}\right)\,cos\left(\frac {\varphi_1 - \varphi_2}{2}\right).
\end{equation}
D.h. die Frequenz ändert sich nicht. Die Phasenverschiebung wird gemittelt und es ergibt sich ein zeitunabhängiger Wichtungsfaktor, der von der Phasendifferenz abhängt. Für $\varphi_1-\varphi_2= \pi $ löschen sich die Signale gegenseitig aus.\\\\
Für eine Phasenverschiebung von $|\varphi_1-\varphi_2|\le \frac{\pi}{2}$ ($\Delta\varphi = \frac{\lambda}{4}$) gilt für den Wichtungsfaktor $cos(\frac{\varphi_1-\varphi_2}{2})\ge 0.7071$ und es liegt konstruktive Interferenz vor.\\ 

\subsubsection*{Beispiel}
Überlagerung zweier Signale mit unterschiedlicher Frequenz:
\begin{align*}
s_1(t) & =cos(\omega_1 t)\\
s_2(t) &=cos(\omega_2 t)
\end{align*}
Mit dem Additionstheorem folgt:
\begin{equation}
s_1(t)+s_2(t)= 2 cos\left(\frac{\omega_1+\omega_2}{2}t\right)\,cos\left(\frac{\omega_1-\omega_2}{2}t\right).
\end{equation}

Es ergibt sich eine Schwebung mit einer hochfrequenten Trägerfrequenz $\frac{\omega_1+\omega_2}{2}$ und einer niederfrequenten Amplitudenmodulation mit der Frequenz $f_A = \frac{1}{2\pi}\frac{\omega_1-\omega_2}{2}$. Interessant ist, dass zu den Zeitpunkten $\frac{n}{2f_A}+\frac{1}{4f_A}, n \in G$ ein Phasensprung auftritt. Die Maxima der Einhüllenden sind ebenfalls $\frac{1}{2f_A}$ entfernt. Beachte: wird eine Fourieranalyse von diesem Signal gemacht, treten nur die Frequenzen $\omega_1$ und $\omega_2$ auf. Die Formulierung mit amplitudenmodulierter Trägerfrequenz ist jediglich äquivalent.\\ 

\subsubsection*{Beispiel}
Ein Signal soll zwei Frequenzen $\omega_1$ und $\omega_2$ enthalten: $s(t) = cos(\omega_1t) + cos(\omega_2t)$. Betrachtet wird die Überlagerung zweier zeitverschobener Signale:
\begin{align*}
s_1(t) &=s(t+t_1)\\
s_2(t) &=s(t+t_2)
\end{align*}

Mit dem Additionstheorem folgt:
\begin{multline}
s_1(t)+s_2(t)= 2 cos\left(\omega_1 t +\omega_1\frac{t_1+t_2}{2}\right)\,cos\left(\omega_1\frac{t_1-t_2}{2}\right)\\
+2 cos\left(\omega_2 t +\omega_2\frac{t_1+t_2}{2}\right)\,cos\left(\omega_2\frac{t_1-t_2}{2}\right).
\end{multline}

\begin{itemize}
\item Die Frequenzen $\omega_1$ und $\omega_2$ bleiben erhalten.
\item $cos(\omega_1\frac{t_1-t_2}{2})=f_1$ und $cos(\omega_2\frac{t_1-t_2}{2})=f_2$ sind zeitunabhängige Wichtungsfaktoren für die beiden Frequenzen.
\item Für $\omega_1<\omega_2$ ist $f_1>f_2$, d.h. die Stapelung ist ein Tiefpass. Die Bedingung für konstruktive Interferenz ist für tiefe Frequenzen eher erfüllt als für hohe Frequenzen. 
\end{itemize}


\section{Beamforming}

Bei Beamforming sucht man die richtige Wellenzahl $\vec{k}$ des Erdbebens, dabei wird die Stappelung von vielen Seismogrammen des gesuchten Bebens durchgeführt. 

Ein Seismogramm $u(x,y,t)$ kann als Funktion des zeitverschobenen Signals und Rauschen dargestellt werden:
\begin{equation}
u(x,y,t) = s(t-t_{j})+m(x_{j},y_{j},t)
\end{equation}
 Dabei ist $s(t-t_{j})$ das zeitverschobene Signal und $m(x_{j},y_{j},t)$ das Rauschen.\\
 Durch Summieren über alle Stationen wird ein Beam $b(t)$ erzeugt :
 
\begin{equation}
 b(t) = \sum_{j=1}^{n} u( x_{j},y_{j},t+t_{j})
\end{equation}
{\small Wobei $j$ der Stationindex ist.}\\\\ 
 Die Slowness sei bekannt. Für Slowness $\vec{p}$ gilt : $ \vec{p}\approx \hat{\vec{p}}$\\
 Die Zeitverschiebung ist: 
 \begin{equation}
 \hat{t}_{j}=\hat{\vec{p}}\hat{r}
 \end{equation}
Alle Seismogramme werden mit einer bestimmten Slowness verschoben. Der Prozess wird für alle Langsamkeiten , $\vec{p}$, durchgeführt. So wird die passende Wellenzahl der ankommenden Wellenfront wird ermittelt. 
 

 \begin{equation}
 b(t) = M_{s}(t)+ \sum_{j=1}^{n} u( x_{j},y_{j},t+t_{j})
 \end{equation}
 dabei ist $M_{s}(t)$ die konstruktive Interferenz und
\[
\sum_{j=1}^{n} u( x_{j},y_{j},t+t_{j})
\]
die destruktive Interferenz.

\subsection{Velocity Spectrum Analysis}
Die Richtung der ankommenden Wellenfront sei bekannt, aber der Betrag der Slowness  $\vert\vec{p}\vert$ und die Phasengeschwindigkeit $C_{x}$ seinen unbekannt.
In diesem Fall könnte die Geschwindigkeit-Spektrum-Analyse (\textit{Velocity Spectrum Analysis}) durchgeführt werden.\\
Dabei wird die Slowness, $|\hat{\vec{p}}|$, systematisch variiert, und jeweils wird ein Beam berechnet.
\[
Stack = f(\vert\hat{\vec{p}}\vert,t)
\]
Alle Hypothese, jede Slowness, wird berechnet und erstellt daraus das Vespagramm. 

\subsection{f-k-Analyse}
Die Frequenz-Wellenzahl Analyse kann auch berechnet werden um den Wellenzahlvektor zu berechnen.
Im Fall, das Geschwindigkeit und Richtung der Wellenfront unbekannt ist, kann eine Frequenz-Wellenzahl-Analyse durchgeführt werden. Dabei wird eine 3D-Fourier-Transformation durchgeführt:
\begin{equation}
u(k_{x},k_{y},\omega) = \iiint_{-\infty}^{\infty} u(x,y,t)\cdot\exp^{(-i(k_{x}x+k_{y}y+\omega t))} dxdyd\omega
\end{equation}
Das Modell (\textbf{für was?}) sei:
\[
u(x_{j},y_{j},\omega) = S(\omega)\cdot\exp^{(-ik\hat{r})}+N(x_{j},y_{j},\omega)
\]
{\small Dabei ist $S(\omega)\cdot\exp({-ik\hat{r}})$ das empfangene Signal, $N(x_{j},y_{j},\omega)$ das Rauschen und $\vec{r}$ ist der Stationsvektor.}\\
Die Hypothese lautet:
\[
\hat{t_{j}} = \hat{\vec{p}} \vec{r}
\]
Stack im Frequenzbereich für alle mögliche $\vec{k}$.
Statt $u(k_{x}, k_{y}, \omega)$ nehmen wir $FK(\hat{\vec{k}},\omega)$ an, so erhalten wir:
\begin{equation}
FK(\hat{\vec{k}},\omega) = M S(\omega)+\sum_{j}N(x_{j},y_{j},\omega)
\end{equation}
Anschließend können $k_{y}$- und $k_{x}$-Komponenten dargestellt werden und aus dem Plot der Wellenzahlvektor $\vec{k}$ abgelesen werden. So zeigt $\vec{k}$ die Richtung der einfallenden Wellenfront.
Weiter ist die Länge des Wellenzahlvektors in relation zur slowness definiert als:
\begin{equation}
\vec{k} = \vec{n} k = \vec{n} \dfrac{\omega}{c_x} = \vec{n} \omega p= \omega \vec{p} 
\end{equation}
Aus der Richtung und der Länge des Wellenzahlvektors $\vec{k}$ berechnet man dann den Slownessvektor $\vec{p}$.\\
Eine wichtige Rolle bei der f-k-Analyse spielt das Auflösungsvermögen eines Arrays. Die Array-Antwort beschreibt die Auflösungsvermögen einers Arrays. Grosse Fläche des Arrays bedeutet eine hohe Auflösung im Wellenzahlbereich. Dagegen eine kleinere Fläche entspricht einer geringeren Auflösung. Bei einer geringen Stationsdichte entsteht bei f-k-Analyse kein klares Maximum im Wellenzahlbereich.
Modell: Im Ortsbereich nimmt man $u(x_{j},y_{j},t) = \delta(t-t_{j})$ an, das ist $u(x_{j},y_{j},t)= \exp^{-i\omega t_{j}}$ im Frequenzbereich. Mit dem Stationsvektor $\vec{r_{j}}$ ist $u(x_{j},y_{j},t) = \exp^{-i\vec{k}\vec{r_{j}}}$  im Frequenzbereich.
Mit dem Verschiebungssatz folgt:
\begin{equation}
FK(\hat{\vec{k}},\omega) = \sum_{j}\exp^{i( \hat{k}-k)\vec{r_{j}}}
\end{equation}
3D-Fourier-Transformation liefert dann eine sinc-Funktion im Wellenzahlbereich. Ein Array, das im Ortsbereich  in $x$-Richtung gestreckt ist, zeigt eine elliptische Auflösung im Wellenzahlbereich.\\
Ein Array, das linear ausgerichtet ist, ist nicht optimal für die Frequenz-Wellenzahl-Analyse. Nur ein Array, das eine Fläche abdeckt zeigt ein eindeutiges Maximum im Wellenzahlbereich.