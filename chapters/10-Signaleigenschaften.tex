\chapter{Signaleigenschaften}

In diesem Kapitel werden Eigenschafen von kausalen und analytischen Signalen abgehandelt.\\

\paragraph{Stabilität} von einem Signal $x(t)$ ist gegeben wenn
\begin{equation}
\int_{-\infty}^\infty |x(t)|^2\,dt < \infty.
\end{equation}

\section{Kausalität eines Signals}
Kausalität eines Signals ist erfüllt wenn:
\[
\begin{split}
f(t) = 0 ,&\:t \leq 0\\
\int_0^\infty f(t)^2\,dt > 0,&\:t > 0
\end{split}
\]
\textit{Kausal} bedeutet hier wenn es sich um eine physikalisch realisierbares System handelt. Also kein Wirkung vor der Ursache stattfindet.

\subsection{Physikalisch Realisierbares System}
Ein physikalisch realisierbares System ist gegeben wenn:
\begin{equation}
\begin{split}
y(t) = f(t) * x(t) & = \int_{-\infty}^\infty f(t')\, x(t-t') dt'\\
& =\int_0^\infty f(t')\, x(t-t') dt'
\end{split}
\end{equation}
Wobei $y(t)$ der Output des Systems $f(t)$ mit dem Input $x(t)$ ist. Weiter ist $t' \geq 0$, d.h. es wird nur auf aktielle und vergangene Werte des Inputs zugegriffen.

\section{Eigenschaften eines Kausalen Systems}
Welche Eigenschaften hat das Spektrum eines kausalen Systems?

\subsection{Hilberttransformation}
Die Hilberttransformation ist eine Phasenverschiebung des Signals um 90$^\circ$. Die passiert zum Beispiel wenn eine Welle durch eine Kaustik, ein Brennpunkt mehrer Wellenpfade, läuft, oder bei der Reflektion an Schichtgrenzen mit Negativer Impendanz.\\
Definition der Hilberttransformation.
\begin{equation}
H\{x(t)\} = \frac{1}{\pi} \int_{-\infty}^\infty \frac{x(t')}{t'-t}\,dt' = x_H(t)
\end{equation}
Problem wenn $t'=t$, dann tritt Singularität auf.\\
Die Lösung ist das Integral im Sinne des Couchy'schen Hauptwerdes zu bilden:
\begin{equation}
H\{x(t)\} = \frac{1}{\pi}\,\lim_{\epsilon \rightarrow 0} \left(\int_{-\infty}^{t-\epsilon} \frac{x(t')}{t'-t}\,dt' + \int_{t+\epsilon}^{\infty} \frac{x(t')}{t'-t}\,dt' \right)
\end{equation}
Wobei für $\epsilon > 0$ gilt.\\\\
Die Rücktransformation der Hilbertransformation ist wie folgt:
\begin{equation}
H^{-1}\{x_H(t)\} = x(t) = -\frac{1}{\pi} \int_{-\infty}^\infty \frac{x_H(t')}{t'-t}\,dt'
\end{equation}

\paragraph{Beispiel der Hilbertransformation} für harmonische Sinusfunktion
\[
\begin{split}
H\{-\sin (\omega_0t)\} = & \cos (\omega_0 t)\\
H\{\cos (\omega_0t)\} = & \sin (\omega_0 t)
\end{split}
\]

\subsubsection*{Hilberttransformation durch Faltung}
Die Hilberttransformation kann durch Faltung dargestellt werden:
\[
H\{x(t)\} = -\frac{1}{\pi t} * x(t)
\]

\subsubsection*{Hilberttransformation im Frequenzbereich}
Die Hiblerttransformation im Frequenzbereich,
\[
F\{H\{x(t)\}\} = i\,\mbox{sign}\,\omega\,X(\omega)
\]
Dabei ist $i\,\mbox{sign}\,\omega = 
\begin{cases}
-i & \omega < 0\\
i & \omega \geq 0
\end{cases}$

\subsubsection*{Berechnung der Hilberttransformation}
Die praktische Berechnung der Hilberttransformation über den analytische Zerlegung des Signals in den Frequenzbereich.
\[
\begin{split}
F\{x(t)\} = & X(\omega)\\
i\,\mbox{sign}\,\omega\,X(\omega) = &F\{H\{x(t)\}\}\\
\end{split}
\]
So ist
\[
F^{-1}\{i\,\mbox{sign}\,\omega\,X(\omega)\} = x_H(t)
\]
die Hilberttransformation des Signals $x(t)$ 